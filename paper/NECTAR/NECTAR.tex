\documentclass[review, 3p,
authoryear]{elsarticle} %review=doublespace preprint=single 5p=2 column
%%% Begin My package additions %%%%%%%%%%%%%%%%%%%

\usepackage[hyphens]{url}

  \journal{17th International NECTAR Conference} % Sets Journal name

\usepackage{graphicx}
%%%%%%%%%%%%%%%% end my additions to header

\usepackage[T1]{fontenc}
\usepackage{lmodern}
\usepackage{amssymb,amsmath}
% TODO: Currently lineno needs to be loaded after amsmath because of conflict
% https://github.com/latex-lineno/lineno/issues/5
\usepackage{lineno} % add
\usepackage{ifxetex,ifluatex}
\usepackage{fixltx2e} % provides \textsubscript
% use upquote if available, for straight quotes in verbatim environments
\IfFileExists{upquote.sty}{\usepackage{upquote}}{}
\ifnum 0\ifxetex 1\fi\ifluatex 1\fi=0 % if pdftex
  \usepackage[utf8]{inputenc}
\else % if luatex or xelatex
  \usepackage{fontspec}
  \ifxetex
    \usepackage{xltxtra,xunicode}
  \fi
  \defaultfontfeatures{Mapping=tex-text,Scale=MatchLowercase}
  \newcommand{\euro}{€}
\fi
% use microtype if available
\IfFileExists{microtype.sty}{\usepackage{microtype}}{}
\usepackage[]{natbib}
\bibliographystyle{elsarticle-harv}

\ifxetex
  \usepackage[setpagesize=false, % page size defined by xetex
              unicode=false, % unicode breaks when used with xetex
              xetex]{hyperref}
\else
  \usepackage[unicode=true]{hyperref}
\fi
\hypersetup{breaklinks=true,
            bookmarks=true,
            pdfauthor={},
            pdftitle={Identifying complexity to reallocate street space: An open-source tool for Portugal},
            colorlinks=false,
            urlcolor=blue,
            linkcolor=magenta,
            pdfborder={0 0 0}}

\setcounter{secnumdepth}{5}
% Pandoc toggle for numbering sections (defaults to be off)


% tightlist command for lists without linebreak
\providecommand{\tightlist}{%
  \setlength{\itemsep}{0pt}\setlength{\parskip}{0pt}}







\begin{document}


\begin{frontmatter}

  \title{Identifying complexity to reallocate street space: An
open-source tool for Portugal}
    \author[CERIS]{Gabriel Valença%
  \corref{cor1}%
  }
   \ead{gabrielvalenca@tecnico.ulisboa.pt} 
    \author[CERIS]{Rosa Félix%
  %
  }
  
    \author[CERIS]{Filipe Moura%
  %
  }
  
    \author[CITUA]{Ana Morais de Sá%
  %
  }
  
      \affiliation[CERIS]{
    organization={CERIS, Instituto Superior Técnico - University of
Lisboa},addressline={Av. Rovisco Pais
1},city={Lisboa},postcode={1049-001},country={Portugal},}
    \affiliation[CITUA]{
    organization={CiTUA, Instituto Superior Técnico - University of
Lisboa},addressline={Av. Rovisco Pais
1},city={Lisboa},postcode={1049-001},country={Portugal},}
    \cortext[cor1]{Corresponding author}
  
  \begin{abstract}
  
  \end{abstract}
  
 \end{frontmatter}

Traditionally, urban road space allocation has relied on the
hierarchical street classification, favoring traffic lanes in arterials
and allocating more space to parking or sidewalks in local streets.
However, a dilemma arises in more complex urban environments that face
limitations in space and must accommodate both mobility and access
functions. Consequently, deciding how much space to allocate in complex
urban areas for these functions is not always evident and requires
tradeoffs. Additionally, these zones tend to have high intensity and
fluctuation of multimodal demands, leading to underutilized spaces at
certain times of the day. There is a potential to reallocate space
dynamically over time according to fluctuations of demand, having a more
efficient and just space distribution. We define a complex space as
facing the mobility vs access dilemma, having high connectivity, having
dense and diverse land use and with high levels of traffic or/and public
transport at least one hour of the day. Zones with these characteristics
tend to have scarce urban space to fulfill the street's mobility and
access functions. To address this issue, we propose a site selection
methodology to identify complex zones within cities on a macro scale
where diverse users and demands compete for space. These zones require a
deeper understanding of urban dynamics to prioritize sustainable
transportation policy. The proposed methodology uses open data such as
road network, information on population, land use, and transit and
traffic dynamics, provided by OpenStreetMaps, national census, Google
Maps API and General Transit Feed Specification (GTFS) sources. We
propose indicators to determine locations that we consider complex to
reallocate road space. In previous work we demonstrated this application
through a case study in Lisbon, offering planners a starting point to
assess activities and temporal-spatial demands when reallocating road
space. We developed an R package that can reproduce the proposed
methodology in any location in Portugal. Adaptations from the initial
methodology were needed due to the different contexts, and scale of
analysis. This methodology could be expanded not only to other countries
(as long as required data exists) but also for applications such as
identifying 30km/h zones.

\bibliography{mybibfile.bib}


\end{document}
